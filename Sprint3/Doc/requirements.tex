% !TEX root = SystemTemplate.tex
\chapter{User Stories, Backlog and Requirements}
\section{Overview}


This section will look at the userstories of the development of the program, the requirements
of the program, the proof o fconcept results, and research task results. It will also look at the
reason for developing this software. 

 The userstories 
Compile source code from tester.   Run compiled code from tester.   Compare two files from tester to see
if they are different.   Traverse a directory looking for test cases.   Write program output to a file.   Write
to human readable log file.   

Below:   list, describe, and define the requirements in this chapter.  
There could be any number of sub-sections to help provide the necessary level of 
detail. 





\subsection{Scope}


This document will contain stakeholder information, 
initial user stories, requirements, proof of concept results, and various research 
task results. 



\subsection{Purpose of the System}
To test a set of basic computer programs written in the C++ language so a grade can quickly be assigned to a class of students.
The system will also be able to generate random test cases from user input.


\section{ Stakeholder Information}


The person most interested in this project is Dr. Logar (the end user) who is looking to quickly, 
and easily, grade programs that are turned in by her CSC 150 students.


\subsection{Customer or End User (Product Owner)}
The product owner on the first sprint was Samuel Carroll, he created a list with his teammates 
and on a day, selected by Dr. Logar, met with her and the other teams' product owner to determine
exatly what Dr. Logar wanted. Samuel was also the team member most involved in keeping the Trello 
board up to date.

The product owner on the second sprint was Erik Hattervig, he gathered the nessary requirments of the
second sprint from Dr Logar, relaied the information to his teammates, and set up the product backlog on
the Trello board.


\subsection{Management or Instructor (Scrum Master)}
The scrum master for the first sprint was Colter Assman, his duties were to ensure that the project 
stayed on track and if any team member ran into some issues he would help them get back on track.   
Colter also was responsible for the running of the daily scrum.

The scrum master for the second sprint was Jonathon Tomes,  he was in charge of managing the scheduling
for the team members, creating spring schedules, and moving tasks from the product backlog to the sprint backlog.
He also lead the scrum meetings.

\subsection{Investors}
There were no investors for our first sprint.


\subsection{Developers --Testers}
Shaun Gruenig was the biggest tester for our program.   As the team technical lead 
he kept us updated on if the project was running as we expected it to, and would 
often debug the issues our code had.

For the second sprint, all three members of the team tested each other's code and gave feed back on bugs and
code quality via Github.


\section{Business Need}
Currently many computer science teachers have to write each test case out by hand.   
This is a very time consuming endeavor (especially considering how many students each
one has), so this program would enable them to write a test cases which will then be input
to our program.   Quickly and accurately giving grades to students.

\section{Requirements and Design Constraints}
The requirements was that our tester run in the Linux environment.   We also needed this 
program to be ready to send out by the time the first CSC 150 program was due, therefore
we only had about two weeks to write and implement this code.

For sprint two we were limited to a set of features that were given to us by Dr. Logar.


\subsection{System  Requirements}
The program must be able to run on a Linux machine, using the GNU operating system.   
therefore the code must able to compile using the GNU compiler.   This means all of our
code must be executable on Linux machines.   Of course we may have had to write this
program for another system, but the Linux environment is the nicest one for us to use.


\subsection{Network Requirements}
No network requirements were needed.


\subsection{Development Environment Requirements}
Linux/GNU system should be able to run our tester. 


\subsection{Project  Management Methodology}
The stakeholders had several requests on how the project was implemented. Including 
what to use to keep track of backlogs and sprint status, which parties had access to the
sprint and produt backlogs, how many sprints will be used for this project, and restrictions
on the source control.
 
\begin{itemize}
\item Trello was used to keep track of the backlogs and sprint status
\item All parties will have access to the Sprint and Product Backlogs
\item Three sprints will encompass this project
\item The sprints will vary in length a little bit but be about 2-3 weeks in general
\item Github was used for source control on the second sprint
\end{itemize}

\section{User Stories}
This section contains information about the user stories (what the program must be able 
to do, and what the user should have to do).



\subsection{Compile and Run Source Code}
The program must be able to compile and run source code found in the directory

For sprint two we now must be able to compile and run test on an entire class directory.

\subsection{Write Pass/Fail and Percentages to Log File} 
This program must be able to write output to a log file and to keep track of the total number
passed cases and the total number of failed cases.

For sprint two we need to be able to keep track of both individual records as well as the whole class.

\subsection{Compare Output with expected output}
The program must be able to compare the output that we get after running a test case to the
output that we expect to get from the test case. The expected output will be found when 
searching the directory.

\subsection{Searching/Traversing the Directory}
The program must be able to search through all the files and sub-directories of the directory 
that we are currently in.

For sprint two this is used in the test directory.

\subsection{Invoking the Program}
The user must be able to run our program by typing `test <directory>'.


\section{Research or Proof of Concept Results}
Most of the code had been written by our team before.   We knew how to run the system 
function in C++ to invoke a system command. We had built a directory crawler in an earlier 
class (though in Windows so some modification had to take place).   All in all starting the 
program we knew we could complete it.

For sprint two much of the same concepts applied for the new features that were added such as
test case generation.

\section{Supporting Material}


In the man pages for the diff function it shows us that it returns one of three values and 
the case those values are returned, a zero if there is no difference between the two files, 
a one if there is a difference between the two files, or a two if something went wrong (doesn't happen often)

