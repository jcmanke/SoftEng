% !TEX root = SystemTemplate.tex


\chapter{Project Overview}
This section provides information regarding the team and their roles,
how the project was managed, and any relatively unkown terminology or
acronyms.



\section{Team Members and Roles}
The team members were Erik Hattervig, the Product Owner; Andrew Koc, Tech Lead;
and Jonathan Tomes, Scrum Master.


\section{Project  Management Approach}

	The program was created through the Scrum Agile Approach.
The sprint length for this project was 2 weeks. We began with a meeting to
decide the user needs and split the program accordingly. Each of us would code
different parts of the program and then we would all test and re-code as needed.
placed on Trello to help design break points to split up the program between team members. 

	The code was stored, backed up, and shared through git hub. The back log and ownership 
was tracked through Trello. The user stories were condensed and 



\section{Phase  Overview}
Sprint 1 (Done by the team We Can't Follow Directions):

This program started in a gathering information phase, the first thing we had to do as a team 
was generate several questions to ask the customer (Dr. Logar) about the software that she 
wanted.   The next phase was asking if it could be done, as most of the code had been written 
at one point or another for our team, we decided that it was a program that could be written.  
The next phase was breaking the program into smaller milestones, we decided that the program 
would need to compile source code, search for test cases, run compiled code with test case, 
output test case results to a file, compare that output file to an expected output file, and then 
write to a file if the two files match or not, if they do we write the text case then passed to a log 
file, if they don't we write the text case and failed to the log file.   Once we are done traversing the 
directory we write the total number of passed, failed and percentages of each.

Sprint 2( Done by the team The Software Engineering Adventure Line):

First we began with an information gathering session by Dr. Logar and the product owners. This is where we
found out the goals of the program and the desired results. After this we were assigned a team's work
from the previous sprint, and evaluted it. Unfortunately we had to do a lot of revising and creating new
functions because the old cold was not very modular. So we divided up the work of reworking the code and
did our parts. After that we gathered and came up with how to meet the new requirements and worked out
what we needed to code, ask the user for, and finished coding the program.

\section{Terminology and Acronyms}
none
