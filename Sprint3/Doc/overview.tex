% !TEX root = SystemTemplate.tex

\chapter{Overview and concept of operations}

This document will look at the Software Engineering Adventure Line team second sprint for building
a program tester. Looking at the team members and their roles, the project
management that we used, the sprint retrospective, any terminology or acronyms
that we use. Next we will look at the user stories, backlog and requirements of the
program. Then we will look at the design and implementation of the program focusing
on major pieces of the code. The next section will look at system and unit testing of 
our program, followed by development environment, and release, setup and
deployment of our program. We will finish this document with a look at a user 
documentation (including a user guide, installation guide and programmer manual)
class index and class documnetation.


\section{Scope}
This document will cover the second sprint of our program tester built for Dr. Logar's
software engineering class in Spring 2014 .


\section{Purpose}
The purpose of this program is to compile, run and test the simple programs created by others. 

The program willl also offer to generate additonal random tests.

\subsection{Normal Run}
The programs it tests are supposed to be guarenteed to compile and run correctly. It will then search through a root directory, find each studen'ts
subdirectory, compile their code into the root directory, open log files for the studen and the class as a whole. Student logs will
go into the student's directory and the class log will be in the root. The student logs will have the results of each test and a final
score. The final score will be based on the precent of the tests passed. Unless the program fails a critcal test. If a student fails a 
test labeled as "crit\_(something).tst", that student will be immediately marked as "FAILED".

\subsection{ Generate Tests}
The program will ask for a number tests to generate, the number of inputs for
each test, and the data type for the tests. The program will then generate a "GeneratedTests" directory in the test directory of the root
folder. At which point, the program will generate the number of tests specified. If the program has previously generated tests, it will
then remove the past generated tests and create new ones. 


\section{Systems Goals}
1) Find student programs to compile.

2) Find the tests and use them to test the found programs.

3) Generate new random test cases.

\section{System Overview and Diagram}
The program will be started on the command line. You may specify a starting directory or not, if not the program will assume that it is running in the
directory it is supposed to be searching in. It will then display a simple menu of Run, which tells the program to run normaly,
searching for student programs and testing them. Or Generate, which will tell the program to generate tests. It will ask
the user for the number of test cases, number of inputs, and the data type to use.   See Figure~\ref{systemdiagram}.

\begin{figure}[tbh]
\begin{center}
\includegraphics[width=0.75\textwidth]{./SystemDiagram}
\end{center}
\caption{System Diagram \label{systemdiagram}}
\end{figure}

\section{Technologies Overview}
The main programing langue used was C++, and was compiled with g++ on a linux environment.
