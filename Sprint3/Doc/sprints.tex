% !TEX root = SystemTemplate.tex

\chapter{Sprint Reports}

\section{Sprint Report \#1}
The testing function can traverse a directory looking for source code. Compile that code, then
look for test cases and run the compiled code with the given test case outputting the result in a 
new document. It can then compare that document with the expected results document and write
to the log file if it passed or failed. Then it will continue looking for test cases until all the test cases
in the root folder have been found. The tester will then write the total number of test cases passed,
total failed and the percentage passed and failed.

\section{Sprint Report \#2}

Had to redo a lot of the code. Most of it wasn't split into seperate functions and some of it was
misdocumented. After getting around that, we got it to travese the root directory, finding the
student directories and testing them against the tests located in the test directory. It can generate random
tests (the number of and data types specified by the user). Introduced a simple menu system to make it easier
to generate then run tests. It will log the results of the testing each student into a student log located in their
directory, and to a class log located in the root directory.

\section{Sprint Report \#3}
