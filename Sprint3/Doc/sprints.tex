% !TEX root = SystemTemplate.tex

\chapter{Sprint Reports}

\section{Sprint Report \#1}
The testing function can traverse a directory looking for source code. Compile that code, then
look for test cases and run the compiled code with the given test case outputting the result in a 
new document. It can then compare that document with the expected results document and write
to the log file if it passed or failed. Then it will continue looking for test cases until all the test cases
in the root folder have been found. The tester will then write the total number of test cases passed,
total failed and the percentage passed and failed.

\section{Sprint Report \#2}

Had to redo a lot of the code. Most of it wasn't split into seperate functions and some of it was
misdocumented. After getting around that, we got it to travese the root directory, finding the
student directories and testing them against the tests located in the test directory. It can generate random
tests (the number of and data types specified by the user). Introduced a simple menu system to make it easier
to generate then run tests. It will log the results of the testing each student into a student log located in their
directory, and to a class log located in the root directory.

\section{Sprint Report \#3}

This section required a long time before we were even able to make it run. We used the test directory supplied on
the website, and the program would register a segmentation fault. It turns out that the previous group had
decided that all tests that were to be run would only be in a directory called tests, that the student directories
would only have one .cpp file and no files that contained cpp, such as a cpp.log. This also applied to the golden.cpp
that was expected to be the only cpp in the root testing directory.

After fighting through these issues, we managed to apply the new functionalities in short order. We implemented a
customized diff function for the new cases, added string generation to the menu, and implemented performance
testing and code coverage statistics through Gcov and Gprof. We also added a way to test if something was looping
for too long, with an option to change the default timeout after selecting to run tests. 
