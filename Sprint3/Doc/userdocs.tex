% !TEX root = SystemTemplate.tex

\chapter{User Documentation}


%\newpage   %% 
%%  The user guide can be an external document which is included here if necessary ...
%%  a single source is the way to go.

\section{User Guide}

To use our program simply run it on the command line. You may specify a starting directory as a command
line argument, or the program will assume to run in the directory it already is in. Next a menu will apear
and ask for one of the options. 

Run will let the program run normaly, it will search through the directory
and find student directories containing the student source code. It will compile the code to the
root directory. It will then search through the test directory in the root to find test casses to run on the
student programs. It will log the result of each test to a student log file located in the student directory, and
the final result will be repeated in an overal class log located in the starting directory. If the student
fails a test labeled "crit\_(something).tst" the student will immediately fail and no more testing will be done.

Generate will then prompt for a number of test cases, the number of inputs per test, and finally a data type
to use for each test. It will then create a sub-directory called "Generated" in the test sub-directory of the root
where the program was prompted to go or started in. If it had previously created tests, it will clear the directory
and create a new one and all new test cases.

Exit will exit the program.


%% \newpage  %%  if needed ...
\section{Installation Guide}

1) Open the terminal and navigate to the directory containing our program source code and the makefile.

2) Run make.


%% \newpage  %%  if needed ...
\section{Programmer Manual}

