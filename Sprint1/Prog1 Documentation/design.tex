% !TEX root = SystemTemplate.tex
\chapter{Design  and Implementation}
This section is used to describe the design details for each of the major components 
in the system.  This section is not brief and requires the necessary detail that 
can be used by the reader to truly understand the architecture and implementation 
details without having to dig into the code.    Sample algorithm: 
 

\section{Compilation}
Please use the included make file to update and rebuild your executable.

\subsection{Technologies  Used}
\begin{itemize}
\item Uses the stdlib.h library to make a system call.
\item C++ vectors
\end{itemize}

\subsection {Phase Overview}
This component has been created for the initial deployment of V1.x

\subsection{Design Details}
\begin{itemize}
\item Find the core name of the file name (trim extension)
\item Construct command: g++ -o corename filename -g
\item Make system call
\end{itemize}

\section{Directory Crawl }

\subsection{Technologies  Used}
\begin {itemize}
\item Uses the dirent.h library.
\item Uses C++ STL vectors
\end {itemize}

\subsection {Component Overview}
This component uses C++ STL vectors to contain the list of test files (denoted by a '.tst' extension) needed to execute against a specified source file.  These are found by making recursive directory searches (see figure 4.1)

\subsection {Phase Overview}
This component has been created for the initial deployment of V1.x

\subsection{Design Details}
\begin{lstlisting}
#include<dirent.h>
#include <string>

void ParseDirectory(string root)
{  
    string temp;

    DIR *dir = opendir(root.c_str()); // open the current directory
    struct dirent *entry;

    if (!dir)
    {
        // not a directory
        return;
    }

    while (entry = readdir(dir)) // notice the single '='
    {
        temp = entry->d_name;
        if ( temp != "." )
        {
            if ( temp != ".." )
            {
                if ( temp[temp.size() - 1] != '~' )
                {
                    int length = temp.length();
                    if ( length > 4 && temp[length-1] == 't' && temp[length-2] 
                        == 's' && temp[length-3] == 't' && temp[length-4] == '.' )
                    {
                        TESTVECTOR.push_back(root+'/'+temp);
                    }
                    else
                    {
                        ParseDirectory(root+'/'+temp);
                    }
                }
            }
        }
    }
    
    closedir(dir);

}
\end{lstlisting}
Figure 4.1, Directory crawling for '.tst' filesS

\section{Test Execution }

\subsection{Technologies  Used}
Uses the stdlib.h library to make system calls.\newline
Uses a vector to hold test cases and their results.

\subsection {Phase Overview}
This component has been created for the initial deployment of V1.x

\subsection{Design Details}
 For each test case:
\begin{itemize}
\item Get input and output file names
\item Execute program against test case
\item Diff output file and expected answer file
\item Determine pass/fail, add to vector
\end{itemize}

\section{Result Logging}

\subsection{Technologies Used}
Uses the ctime library to generate test run timestamps.\newline
Uses a vector to store test results.

\subsection{Design Details} 
\begin{lstlisting}
#include <ctime>
#include <vector>

void WriteLog(string prog)
{
    // make name / date log.
    time_t now;
    time(&now);
    string currTime = ctime(&now);
    string name = prog + " " + currTime.substr(0,currTime.length() - 2) + ".log";
    ofstream log(name.c_str());

    for (int i = 0; i < TESTVECTOR.size(); i+=1)
    {
        log << TESTVECTOR[i] << endl; // flush buffer as long strings
    }

    // now push stats
    log << "            |" << endl;
    log << "Bottom Line V" << endl;
    log << "--------------------------------------------------------------------------------" << endl;

    log << "Correct: " << CORRECTTESTS << "\nFailed: " << TESTVECTOR.size() - CORRECTTESTS << endl;
    log << "Success Rate: " << CORRECTTESTS/ TESTVECTOR.size() * 100 << "%" << endl; 
}
\end{lstlisting}