% !TEX root = SystemTemplate.tex

\chapter{User Documentation}

This section will contain the basis for any end user documentation for the system, 
and will cover the basic steps for the setup and use of the system.


%\newpage   %% 
%%  The user guide can be an external document which is included here if necessary ...
%%  a single source is the way to go.

\section{User Guide}

Usage of the Auto Tester is primarily concerned with the format and placement of the test case files, placement of truth program and programs to test. Test case files must be located in the exeution directory or below.  The truth program needs to be placed at the level of execution and all programs to test must be stored in sub-directories.

\subsection{Test Case Files}
Files that contain the test cases themselves need to be given a .tst extension. The filname proceeding the extension will be used as the name of that test in both the local and summary log files. The contents of the file will be the raw input to the student's program.

\subsection{Expected Output Files}
For each test case file, there needs to be a corresponding file that contains the expected output. This file must have the same name as the test case, but instead have a .ans extension.

\subsection{Results}
The results will be placed in a file with a .log extension. The filename will be timestamped and contain the name of the student program.  A final log will be written to the execution directory with the name log and the current date / time.

%% \newpage  %%  if needed ...
\section{Installation Guide}
If running from the .cpp file, the system will first need to be compiled. The user will need to be in the 
same directory as the .cpp file with a Linux command line terminal, and use the following command: \\
g++ -o tester tester.cpp or make

After this is completed, the user should have an
executable file they can use by invoking the following: \\
./tester


%% \newpage  %%  if needed ...
\section{Programmer Manual}
The code contained in the .cpp file is written in c++ and is to be compiled with gcc. 

